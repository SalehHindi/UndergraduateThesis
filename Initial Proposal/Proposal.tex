\documentclass{article}

\usepackage{amssymb}
\usepackage{amsmath}
\usepackage{graphicx}
\usepackage{mathptmx}
\usepackage{lipsum}  % for sample text
\usepackage[T1]{fontenc}
\usepackage{textcomp}
\usepackage{dirtytalk}

%include these lines if you want to use the LaTeX "theorem" environments
\newtheorem{theorem}{Problem}[section]
\newtheorem{definition}[theorem]{Definition}
\newtheorem{lemma}[theorem]{Lemma}
\newtheorem{corollary}[theorem]{Corollary}


%include lines like this if you want to define your own commands 
%to save typing
\newcommand{\PROOF}{\noindent {\bf Proof}: }
\newcommand{\REF}[1]{[\ref{#1}]}
\newcommand{\Ref}[1]{(\ref{#1})}
\newcommand{\dt}{\mbox{\rm   dt}}
\newcommand{\phat}{\hat{p}}


 \usepackage{setspace}
 \onehalfspacing
 \newtheorem{mytheorem}{Theorem}
 \numberwithin{mytheorem}{subsection} % important bit


\begin{document}

	\title{Thesis Proposal}
	\author{Saleh Hindi}

	\maketitle

	\section{Markov Chains and Intransitive Games}
	I want to study Markov Chains for my capstone project because the subject combines graph theory, probability, and linear algebra so it seems like a fitting stop to my undergraduate math degree. Specifically, I want to explore the Penney Ante problem which can be stated as 
	
	\begin{theorem}[Penney Ante Problem]
	\label{penney}
		Given a sequence $S_n \in {S_0, s_1, s_2, \dots, S_n}$ of countable length $n$, what is $E(S_i)$ the expected waiting time for $S_i$ to appear? Furthermore, what is $P(S_i)$, where $P(S_i)$ is the probability that $S_i$ will appear first? Does $P(S_i) = P(S_j)$ imply $E(S_i) = E(S_j)$? 
	\end{theorem}
	
	This is the martingale approach to the problem discovered by the eponymous Walter Penny  \cite{gardner}. Nickerson, Gardner, and Breen describe Penney's problem as given two sequences of length $n-1$ coin flips, $S_i = HHHHH$ and $S_j = THTHHT$, which coin flip is expected to come first in a series of random coin flips? What is the expected waiting time for these two sequences? As it turns out although $E(S_i) > E(S_j)$, $P(S_i) > P(S_j)$ \cite{nickerson} \cite{gardner} \cite{breen}. This counterintuitive result fascinated my attention. This problem could also be formulated in terms of a sequence of rolls of a die with similarly defined waiting times and first appearence probabilies \cite{li}.    \\

	From this formulation, Li and Wendell develop formulas for the stopping time of $S_i$, $E(S_i)$ and for the probability that $S_i$ appears before $S_j$ \cite{li}. While Li uses a martingale approach, Wendel bases his approach in Markov chains. \cite{wendel} \\

    The result of the Penney Ante problem is that for the sequences $A =  HHHHH$ and $B = HTHHT$, the expected waiting time (in tosses) of A is greater than the expected waiting time of B but the probability of seeing A first is also greater than the probability of seeing B first \cite{nickerson}. This problem has applications to biology as a DNA sequence is a sequence of nucleotides represented by letters A, T, C, and G \cite{breen}. This problem also has application in computer science as the Boyr-Moore string matching algorithm is based on this result \cite{breen}. I will attempt to study properties of this game including how $E(S_i)$ varies with $n$ and the number of choices for each $n$. I will also explore if there other games with this counterintuitive result.

	\begin{thebibliography}{1}
		\bibitem{breen}
			Breen, Stephen, Waterman Michael S., and Zhang Ning. "Renewal Theory for Several Patterns." Journal of Applied Probability 22.1 (1985): 228-34. Web.
		\bibitem{gardner}
			Gardner, Martin. "Mathematical Games: On the Paradoxical Situations That Arise from Nontransitive Relations." Scientific American 10 (1974): 120-25. Print.
		\bibitem{li}
			Li, Shuo-Yen Robert. "A Martingale Approach to the Study of Occurrence of Sequence Patterns in Repeated Experiments." The Annals of Probability 8.6 (1980): 1171-176. Web.
		\bibitem{nickerson}
			Nickerson, R. S. "Penney Ante: Counterintuitive Probabilities in Coin Tossing." The UMAP Journal 28.4 (2007): 503-32. JSTOR. Web. 8 Sept. 2016. 
		\bibitem{wendel}
			L.J Guibas, A.M Odlyzko, String overlaps, pattern matching, and nontransitive games, Journal of Combinatorial Theory, Series A, Volume 30, Issue 2, 1981, Pages 183-208, ISSN 0097-3165, http://dx.doi.org/10.1016/0097-3165(81)90005-4.
			(http://www.sciencedirect.com/science/article/pii/0097316581900054)	

	\end{thebibliography}

	\section{Elliptic Curves}
		Another consideration for my thesis involves studying the Elliptic Curves, specifically the Sato Tate conjecture on elliptic curves. Let $E := y^2 = x^3 + ax + b$ for $a, b \in \mathbb{R}$ be the elliptic curve and define $E_p$ to be $E_p:= E \mod p$ for any prime $p$. $E_p$ is called the elliptic curve over finite field with $p$ elements \cite{sutherland}. We can define an addition operation on any points $P_1, P_2$ on the elliptic curve. Let $P_1 + P_2 = P_3$ where $P_3$ is the point found by finding $P_3'$, the point found by intersecting the line through $P_1$ and $P_2$ and the elliptic curve. The third point $P_3$ is the reflection of $P_3'$ over the x-axis. This is not the same as addition of coordinate points on the plane \cite{thebasictheory}. However, this operation turns the set of points in $E$ into an \say{additive abelian group with $\infty$ as the identity element} \cite{thebasictheory}. \\

		One idea for my paper is to explore properties of subsets of the curves, like $E_p$ over the complex numbers or rational numbers. Sutherland references some properties of $E_p$ like the Trace of Frobenius. Let $\mathbb{F}_p$ be a finite field and let the \say{trace of Frobenius} be defined by $a_p : = p + 1 - \#E_p(\mathbb{F}_p)$ \cite{sutherland}. What happens to $a_p$ as $p$ varies? There is also possibility of exploring geometric properties of $E$ and $E_p$. Are any property of the elliptic curve invariant under isomorphism or endomorphism?

	\begin{thebibliography}{1}
		\bibitem{sutherland}
			Sutherland, Andrew. "Sato-Tate Distributions." Arizona Winter School 2016. Course notes.

		\bibitem{euclid}
			Clozel, L. "The Sato-Tate Conjecture." Current Developments in Mathematics 2006.1 (2006): 1-34. JSTOR. Web. 7 Sept. 2016. 

		\bibitem{baier}
			Baier, Stephan. "The Sato-Tate Conjecture on Average for Small Angles." Transactions of the American Mathematical Society 361.4 (2009): 1811-832. JSTOR. Web. 8 Sept. 2016.

		\bibitem{thebasictheory}
			 Washington, Lawrence C. Elliptic Curves Number Theory and Cryptography. Boca Raton: CRC, 2008. Print. 

	\end{thebibliography}

	\section*{My Thesis}
		I have thought long and hard about what I want my thesis to be. It's going to be my final project at Haverford so I want to make it something I am proud of and so I can present to people outside of the college. I want to end my last year on a high note. That said, my first choice is the Penney Ante problem because I have not gotten a chance to explore very much of either combinatorics or probability at Haverford. As I said, the mix of linear algebra and probability is particularly exciting to me and I think this topic relates to my personal interests. The topics presented best match the research interests of Curtis Greene and Heidi Goodson so I prefer to work with one of these two professors. I would be elated to work with either faculty member so I will be happy with any choice the department makes.

\end{document}

